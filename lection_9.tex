% !TeX spellcheck = russian_english
% !TeX encoding = UTF-8
\documentclass[a4paper, fleqn]{article}

\usepackage{indentfirst} % Красная строка
\usepackage[T2A]{fontenc} % Поддержка русских букв
\usepackage[utf8]{inputenc} % Кодировка utf8
\usepackage[russian]{babel} % руссификация
\usepackage{amssymb} % дополнительные символы
\usepackage{textcomp} % дополнительные текстовые символы
\usepackage{amsmath} % матрицы
%\usepackage{pdfpages}
%\usepackage{pgfplots}
%\usepackage{pgfplotstable}
\usepackage{listings}
% листинги
\lstset{language=Python, tabsize=4, language=Python}


\textwidth = 16 cm
\oddsidemargin= 0 cm
\evensidemargin= 1 cm


% \abovedisplayskip = 0 pt %.2\abovedisplayskip
% \belowdisplayskip = .2\belowdisplayskip
% \abovedisplayshortskip=.2\abovedisplayshortskip
% \belowdisplayshortskip=.2\belowdisplayshortskip
% \topsep= 0 pt

% \setlength{\mathindent}{1.2 cm}
% \setlength{\topsep}{0 pt}
% \setlength{\abovedisplayskip}{0 pt}

% \clubpenalty = 5000 % запрет висячих строк
% \widowpenalty = 5000
\binoppenalty=10000
\relpenalty=10000

% собственные команды и окружения

\newenvironment{example}[1][]{\medskip \noindent \textbf{Пример. #1}\par \nopagebreak}{\medskip \par} % окружение-"пример"


\title{Лекция \textnumero\,9}
% {\huge \vspace{3 cm}}}

\author{Т.\,Ф. Хирьянов}

\date{}

\begin{document}
	\maketitle
	
\subsection*{Рекурсия}
% мой листинг
\ttfamily{
	\begin{tabbing}
		\hspace{8mm}\=\hspace{8mm}\=\hspace{8mm}\=\hspace{3cm}\=\kill
		
	\end{tabbing}
}


% 09:20

% 18:00
\ttfamily{
	\begin{tabbing}
		\hspace{8mm}\=\hspace{8mm}\=\hspace{8mm}\=\hspace{3cm}\=\kill
		def make\_matroska(size, number):\\
		\> if number > 1:\\
		\> \> print('низ размера', size)\\
		\> \> make\_matroska(size - 1, number - 1)\\
		\> \> print('верх размера', size)\\
		\> else:\\
		\> \> print('матрешечка размера', size)\\
	\end{tabbing}
}
\subsection*{Поиск НОД. Алгоритм Евклида}	
% 21:00

% 24:50
\ttfamily{
	\begin{tabbing}
		\hspace{8mm}\=\hspace{8mm}\=\hspace{8mm}\=\hspace{3cm}\=\kill
		def my\_gcd(a, b):\\
		\> if b == 0:\\
		\> \> return a\\
		\> else:\\
		\> \> return my\_gcd(b, a\%b)\\
	\end{tabbing}
}

\subsection*{Факториал}
% 26:30

% 27:40
\ttfamily{
	\begin{tabbing}
		\hspace{8mm}\=\hspace{8mm}\=\hspace{8mm}\=\hspace{3cm}\=\kill
		def factor(n):\\
		\> return 1 if n == 0 else factor(n - 1)*n\\
	\end{tabbing}
}

\subsection*{Числа Фибоначчи}
% 00:00

% 01:50
\ttfamily{
	\begin{tabbing}
		\hspace{8mm}\=\hspace{8mm}\=\hspace{8mm}\=\hspace{3cm}\=\kill
		def fib(n):\\
		\> if n < 2:\\
		\> \> return n\\
		\> else:\\
		\> \> return fib(n - 1) + fib(n - 2)
	\end{tabbing}
}

\subsection*{Быстрое возведение в степень}	
% 07:00

% 11:50
\ttfamily{
	\begin{tabbing}
		\hspace{8mm}\=\hspace{8mm}\=\hspace{8mm}\=\hspace{3cm}\=\kill
		def fast\_power(a, n):\\
		\> if n == 0:\\
		\> \> return 1\\
		\> elif n\%2 == 1:\\
		\> \> return a*fast\_power(a, n - 1)\\
		\> else: \> \> \> n\%2 == 0\\
		\> \> return fast\_power(a*a, n//2)\\
	\end{tabbing}
}

\subsection*{Ханойские башни}
% 13:00

% 21:50
\ttfamily{
	\begin{tabbing}
		\hspace{8mm}\=\hspace{8mm}\=\hspace{8mm}\=\hspace{3cm}\=\kill
		def hanoi(n, i=1, j=2):\\
		\> if n == 1:\\
		\> \> print('переставить 1', 'блин с', i, 'на', j, 'стержень')\\
		\> else:\\
		\> \> hanoi(n - 1, i, 6 - i - j)\\
		\> \> print('переставить', n, 'блин', n - 1, i, 'на', j, 'стержень')\\
		\> \> hanoi(n - 1, 6 - i - j, j)\\
	\end{tabbing}
}

\subsection*{Генерация комбинаторных объектов}
% 23:30

% 28:00
\ttfamily{
	\begin{tabbing}
		\hspace{8mm}\=\hspace{8mm}\=\hspace{8mm}\=\hspace{3cm}\=\kill
		def bin\_gen(n, prefix = ''):\\
		\> if n == 0:\\
		\> \> print(prefix)\\
		\> else:\\
		\> \> assert(n > 0)\\
		\> \> bin\_gen(n - 1, prefix + '0')\\
		\> \> bin\_gen(n - 1, prefix + '1')\\
	\end{tabbing}
}

% 34:40
\ttfamily{
	\begin{tabbing}
		\hspace{8mm}\=\hspace{8mm}\=\hspace{8mm}\=\hspace{8mm}\=\hspace{3cm}\=\kill
		def perestanov\_gen(n, A=[])\\
		\> if len(A) == n:\\
		\> \> print(*A)\\
		\> else:\\
		\> \> for x in range(1, n + 1):\\
		\> \> \> if x not in A:\\
		\> \> \> \> perestanov\_gen(n, A + [x])\\
	\end{tabbing}
}


\end{document}