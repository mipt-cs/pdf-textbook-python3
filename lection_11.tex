% !TeX spellcheck = russian_english
% !TeX encoding = UTF-8
\documentclass[a4paper, fleqn]{article}

\usepackage{indentfirst} % Красная строка
\usepackage[T2A]{fontenc} % Поддержка русских букв
\usepackage[utf8]{inputenc} % Кодировка utf8
\usepackage[russian]{babel} % руссификация
\usepackage{amssymb} % дополнительные символы
\usepackage{textcomp} % дополнительные текстовые символы
\usepackage{amsmath} % матрицы
%\usepackage{pdfpages}
%\usepackage{pgfplots}
%\usepackage{pgfplotstable}
\usepackage{listings}
% листинги
\lstset{language=Python, tabsize=4, language=Python}


\textwidth = 16 cm
\oddsidemargin= 0 cm
\evensidemargin= 1 cm


% \abovedisplayskip = 0 pt %.2\abovedisplayskip
% \belowdisplayskip = .2\belowdisplayskip
% \abovedisplayshortskip=.2\abovedisplayshortskip
% \belowdisplayshortskip=.2\belowdisplayshortskip
% \topsep= 0 pt

% \setlength{\mathindent}{1.2 cm}
% \setlength{\topsep}{0 pt}
% \setlength{\abovedisplayskip}{0 pt}

% \clubpenalty = 5000 % запрет висячих строк
% \widowpenalty = 5000
\binoppenalty=10000
\relpenalty=10000

% собственные команды и окружения

\newenvironment{example}[1][]{\medskip \noindent \textbf{Пример. #1}\par \nopagebreak}{\medskip \par} % окружение-"пример"


\title{Лекция \textnumero\,11}
% {\huge \vspace{3 cm}}}

\author{Т.\,Ф. Хирьянов}

\date{}

\begin{document}
	\maketitle

\subsection*{Быстрая сортировка Хоара}

% 3:30


\texttt{
	\begin{tabbing}
		\hspace{8mm}\=\hspace{8mm}\=\hspace{8mm}\=\hspace{3cm}\=\kill
		from random import choice\\
		def hoar\_sort(A):\\
		\> if len(A) <= 1:\\
		\> \> return A\\
		\> barrier = choice(A)\\
		\> left = [x for x in A if x < barrier]\\
		\> middle = [x for x in A x == barrier]\\
		\> right = [x for x in A x > barrier]\\
		\> left = hoar\_sort(left)\\
		\> right = hoar\_sort(right)\\
		\> return left + middle + right
	\end{tabbing}
}


\subsection*{Быстрая сортировка слиянием}

% 27:30

\texttt{
	\begin{tabbing}
		\hspace{8mm}\=\hspace{8mm}\=\hspace{8mm}\=\hspace{3cm}\=\kill
		def merge(A, B):\\
		\> Res = []\\
		\> i = 0\\
		\> j = 0\\
		\> while i < len(A) and j < len(B):\\
		\> \>if A[i] < B[j]:\\
		\> \> \> Res.append(A[i])\\
		\> \> \> i += 1\\
		\> \> else:\\
		\> \> \> Res.append(B[j])\\
		\> \> \> j += 1\\
		\> Res += A[i:] + B[j:]\\
		\> return Res
	\end{tabbing}
}

\texttt{
	\begin{tabbing}
		\hspace{8mm}\=\hspace{8mm}\=\hspace{8mm}\=\hspace{3cm}\=\kill
		def merge\_sort(A):\\
		\> if len(A) <= 1:\\
		\> \> return A\\
		\> left = A[:len(A) // 2]\\
		\> right = A[len(A) // 2:]\\
		\> left = merge\_sort(left)\\
		\> right = merge\_sort(right)\\
		\> return merge(left, right)
	\end{tabbing}
}

\subsection*{Пирамидальная сортировка}

% 52:30

\texttt{
	\begin{tabbing}
		\hspace{8mm}\=\hspace{8mm}\=\hspace{8mm}\=\hspace{3cm}\=\kill
		from heapq import*
		def heap\_sort(A):\\
		\> heapify(A)\\
		\> Res = []\\
		\> while len(A) != 0:\\
		\> \> x = heappop(A)\\
		\> \> Res.append(x)\\
		\> return Res
	\end{tabbing}
}

\subsection*{Устойчивость сортировки}

% 1:08:00



\end{document}