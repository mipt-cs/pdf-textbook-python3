% !TeX spellcheck = russian_english
% !TeX encoding = UTF-8
\documentclass[a4paper, fleqn]{article}

\usepackage{indentfirst} % Красная строка
\usepackage[T2A]{fontenc} % Поддержка русских букв
\usepackage[utf8]{inputenc} % Кодировка utf8
\usepackage[russian]{babel} % руссификация
\usepackage{amssymb} % дополнительные символы
\usepackage{textcomp} % дополнительные текстовые символы
\usepackage{amsmath} % матрицы
%\usepackage{pdfpages}
%\usepackage{pgfplots}
%\usepackage{pgfplotstable}
\usepackage{listings}
% листинги
\lstset{language=Python, tabsize=4, language=Python}


\textwidth = 16 cm
\oddsidemargin= 0 cm
\evensidemargin= 1 cm


% \abovedisplayskip = 0 pt %.2\abovedisplayskip
% \belowdisplayskip = .2\belowdisplayskip
% \abovedisplayshortskip=.2\abovedisplayshortskip
% \belowdisplayshortskip=.2\belowdisplayshortskip
% \topsep= 0 pt

% \setlength{\mathindent}{1.2 cm}
% \setlength{\topsep}{0 pt}
% \setlength{\abovedisplayskip}{0 pt}

% \clubpenalty = 5000 % запрет висячих строк
% \widowpenalty = 5000
\binoppenalty=10000
\relpenalty=10000

% собственные команды и окружения

\newenvironment{example}[1][]{\medskip \noindent \textbf{Пример. #1}\par \nopagebreak}{\medskip \par} % окружение-"пример"


\title{Лекция \textnumero\,8}
% {\huge \vspace{3 cm}}}

\author{Т.\,Ф. Хирьянов}

\date{}

\begin{document}
	\maketitle
% ghbvth	
\texttt{
	\begin{tabbing}
		\hspace{8mm}\=\hspace{8mm}\=\hspace{8mm}\=\hspace{3cm}\=\kill
		def fib(n):\\
		\> F = [0, 1] + [0]*(n - 1)\\
		\> for i in range(2, n + 1):\\
		\> \> F[i] = F[i - 1] + F[i - 2]\\
		\> return F[n]
	\end{tabbing}
}	

\subsection*{Сортировка выбором}	
	
	\texttt{
		\begin{tabbing}
			\hspace{8mm}\=\hspace{8mm}\=\hspace{8mm}\=\hspace{3cm}\=\kill
			for pos in range(N - 1):\\
			\> for i in range(pos + 1, N):\\
			\> \> if A[i] < A[pos]:\\
			\> \> \> tmp = A[i]\\
			\> \> \> A[i] = A[pos]\\
			\> \> \> A[pos] = tmp
		\end{tabbing}
	}	
	
\subsection*{Сортировка методом пузырька}

\texttt{
	\begin{tabbing}
		\hspace{8mm}\=\hspace{8mm}\=\hspace{8mm}\=\hspace{3cm}\=\kill
		for prohod in range(1, N):\\
		\> for i in range(N - prohod):\\
		\> \> if A[i] > A[i + 1]:\\
		\> \> \> tmp = A[i]\\
		\> \> \> A[i] = A[i + 1]\\
		\> \> \> A[i + 1] = tmp
	\end{tabbing}
}

\subsection*{Сортировка дурака}	
	
	\texttt{
		\begin{tabbing}
			\hspace{8mm}\=\hspace{8mm}\=\hspace{8mm}\=\hspace{3cm}\=\kill
			i = 0\\
			while i < N - 1:\\
			\> if A[i] < A[i + 1]:\\
			\> \> i += 1\\
			\> else:\\
			\> \> tmp = A[i]\\
			\> \> A[i] = A[i - 1]\\
			\> \> A[i + 1] = tmp\\
			\> \> i = 0
			\end{tabbing}
			}

\subsection*{Сортировка подсчетом}
	
	% 33:00
	\texttt{
		\begin{tabbing}
			\hspace{8mm}\=\hspace{8mm}\=\hspace{8mm}\=\hspace{3cm}\=\kill
			frequency = [0]*10\\
			digit = int(input())\\
			while 0 <= digit <= 9:\\
			\> frequency[digit] += 1\\
			\> digit = int(input())\\
			for digit in range(10):\\
			\> print(*[digit]*frequency[digit])
		\end{tabbing}
	}
\subsubsection*{Проверка типа входных данных}

\texttt{
	\begin{tabbing}
		\hspace{8mm}\=\hspace{8mm}\=\hspace{8mm}\=\hspace{3cm}\=\kill
		def sort\_number(A):\\
		\> assert(type(A[0]) == 'class<int>')
	\end{tabbing}
}

\subsection*{Поразрядная сортировка}

Только для целых чисел или коротких строк.

\texttt{
	\begin{tabbing}
		\hspace{8mm}\=\hspace{8mm}\=\hspace{8mm}\=\hspace{3cm}\=\kill
		A = [int(x) for x in input().split()]\\
		max\_num\_len = max([len(str(x)) for x in A])\\
		for radix in range(0, max\_num\_len):\\
		\> B = [[] for i in range(10)]\\
		\> for x in A:\\
		\> \> digit = (x // (10 ** radix)) \% 10\\
		\> \> B[digit].append(x)\\
		\> \> A[:] = []\\
		\> for digit in range(10):\\
		\> A += B[digit]
	\end{tabbing}
}

\end{document}