% !TeX spellcheck = russian_english
% !TeX encoding = UTF-8
\documentclass[a4paper, fleqn]{article}

\usepackage{indentfirst} % Красная строка
\usepackage[T2A]{fontenc} % Поддержка русских букв
\usepackage[utf8]{inputenc} % Кодировка utf8
\usepackage[russian]{babel} % руссификация
\usepackage{amssymb} % дополнительные символы
\usepackage{textcomp} % дополнительные текстовые символы
\usepackage{amsmath} % матрицы
\usepackage{pdfpages}
\usepackage{pgfplots}
\usepackage{pgfplotstable}
%\ifx\pdfoutput\undefined
%\usepackage{graphicx}
%\else
%\usepackage[pdftex]{graphicx}
%\fi

\textwidth = 16 cm
\oddsidemargin= 0 cm
\evensidemargin= 1 cm


% \abovedisplayskip = 0 pt %.2\abovedisplayskip
% \belowdisplayskip = .2\belowdisplayskip
% \abovedisplayshortskip=.2\abovedisplayshortskip
% \belowdisplayshortskip=.2\belowdisplayshortskip
% \topsep= 0 pt

\setlength{\mathindent}{1.2 cm}
% \setlength{\topsep}{0 pt}
% \setlength{\abovedisplayskip}{0 pt}
\pgfplotsset{major grid style={very thick}}
\pgfplotsset{myaxis/.style = {title=Мой рисунок,
	minor tick num = 4,
	xlabel={$x$,\,м},
	ylabel={$y$,\,A},
	grid={both},
	axis x line=middle,
	axis y line=middle,
	scaled ticks=true,
	legend pos = outer north east}}
%\pgfplotsset{regression/.style = {black, solid, mark=none, y={create col/linear regression={y}}}}

% \clubpenalty = 5000 % запрет висячих строк
% \widowpenalty = 5000
\binoppenalty=10000
\relpenalty=10000

% собственные команды и окружения

% \newcommand*{\spt}[2][ени]{простран\-ств#2-врем#1}
% \newcommand*{\ks}[1]{$K$-систем#1} % учитывать пробел!
% \newcommand*{\kss}[1]{$K'$-систем#1} % учитывать пробел!
\newenvironment{example}[1][]{\medskip \noindent \textbf{Пример. #1}\par \nopagebreak}{\medskip \par} % окружение-"пример"


\title{Лекция \textnumero\,2}
	% {\huge \vspace{3 cm}}}

\author{}

\date{}

\begin{document}
	\maketitle
	\subsection*{Преимущества и недостатки языка программирования Python}
		Достоинства:
		\begin{enumerate}
			\item Современность.
			
			Встроенно очень много удобных методик программирования.
			
			\item Универсальность.
			
			На этом языке можно программировать любое приложение от скрипта операционной системы до игры на мобильном телефоне.
			
			\item Богатая стандартная библиотека.
			
			В ней предусмотренно огромное количество функций, включая работу с сетями и математическими выражениями.
			
			\item Кроссплатформенность
			
			Интерпретатор Python может работать в любой операционной системе, на компьютерах с разной архитектурой.
			
			\item Интерпретируемость
			
			Одним из следствий интерпретируемости является то, что в переменную можно сохранять данные разных форматов. 
			
			\begin{example}
				$x = 123$  \# целое число \\
				$x = 'python'$  \# строка	
			\end{example}
		\end{enumerate}
		
	\subsection* {Ссылочная модель данных в Python}
	
		В Python нет операции присваивания. Запись\\
		$x = 123$ \\
		означает, что объект 123 связывается с ссылкой x. А сама операция является связыванием объекта и ссылки.
		Кусок кода \\
		$x = int(x)$ \\
		будет выполняться следующим образом. Сначала выполнится выражение стоящее справа, затем порожденный им объект, сохранится в некоторой области памяти, вообще говоря, отличающейся от того участка, на который указывал $x$ ранее. В заключении ссылка $x$ связывается с этим новым объектом.
		
		При этом старый объект удаляется сборщиком мусора, если на него больше нет ссылок.
		
	\subsection*{Отличия языков программирования Python2 и Python3}
	
		При развитии и улучшении языков программирования часто бывает необходимо кардинально изменить концепции привычных вещей. При таком переходе теряется совместимость старых и новых версий языка. Так произошло с Python2 и Python3. 
		
		\begin{example}
			
			функция input() в этих версиях языка ведет себя по разному. Так в Python2 выражение\\
			$x = input('5 + 3')$\\
			вычислит значение суммы и сохранит его в $x$.
			В Python3 выражение это же выражение выведет на экран\\
			$5 + 3$\\
			а затем считает данные с клавиатуры и сохранит их в $x$ в виде строки.
			
		\end{example}
		
\end{document}