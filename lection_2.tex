% !TeX spellcheck = russian_english
% !TeX encoding = UTF-8
\documentclass[a4paper, fleqn]{article}

\usepackage{indentfirst} % Красная строка
\usepackage[T2A]{fontenc} % Поддержка русских букв
\usepackage[utf8]{inputenc} % Кодировка utf8
\usepackage[russian]{babel} % руссификация
\usepackage{amssymb} % дополнительные символы
\usepackage{textcomp} % дополнительные текстовые символы
\usepackage{amsmath} % матрицы
%\usepackage{pdfpages}
%\usepackage{pgfplots}
%\usepackage{pgfplotstable}
\usepackage{listings}
% листинги
\lstset{language=Python, tabsize=4, language=Python}


\textwidth = 16 cm
\oddsidemargin= 0 cm
\evensidemargin= 1 cm


% \abovedisplayskip = 0 pt %.2\abovedisplayskip
% \belowdisplayskip = .2\belowdisplayskip
% \abovedisplayshortskip=.2\abovedisplayshortskip
% \belowdisplayshortskip=.2\belowdisplayshortskip
% \topsep= 0 pt

% \setlength{\mathindent}{1.2 cm}
% \setlength{\topsep}{0 pt}
% \setlength{\abovedisplayskip}{0 pt}

% \clubpenalty = 5000 % запрет висячих строк
% \widowpenalty = 5000
\binoppenalty=10000
\relpenalty=10000

% собственные команды и окружения

\newenvironment{example}[1][]{\medskip \noindent \textbf{Пример. #1}\par \nopagebreak}{\medskip \par} % окружение-"пример"


\title{Лекция \textnumero\,2}
% {\huge \vspace{3 cm}}}

\author{Т.\,Ф. Хирьянов}

\date{}

\begin{document}
	\maketitle
	\subsection*{Преимущества языка программирования Python}
		
		\begin{enumerate}
			\item \emph{Современность.}\\
			Встроенно очень много удобных методик программирования.
			
			\item \emph{Универсальность.}\\			
			На этом языке можно программировать любое приложение от скрипта операционной системы до игры на мобильном телефоне.
			
			\item \emph{Богатая стандартная библиотека.}\\			
			В ней предусмотренно огромное количество функций, включая работу с сетями и математическими выражениями.
			
			\item \emph{Кроссплатформенность}\\			
			Интерпретатор Python может работать в любой операционной системе, на компьютерах с разной архитектурой.
			
			\item \emph{Интерпретируемость}\\			
			Одним из следствий интерпретируемости является то, что в переменную можно сохранять данные разных форматов. 
			\texttt{
				\begin{tabbing}
					\hspace{8mm}\=\hspace{2cm}\=\kill
					x = 123		
					x = "python" 	
				\end{tabbing}
			}
			Так 123 -- это целое число, а python -- строка.
		\end{enumerate}
		% указать недостатки
		
	\subsection* {Ссылочная модель данных в Python. Динамическая~типизация.}
	
		В Python нет операции присваивания. Запись
		\texttt{
			\begin{tabbing}
				\hspace{8mm}\=\hspace{2cm}\=\kill
				x = 123	
			\end{tabbing}
		}
		означает, что объект 123 связывается с ссылкой x. А сама операция является связыванием объекта и ссылки.
		Кусок кода
		\texttt{
			\begin{tabbing}
				\hspace{8mm}\=\hspace{2cm}\=\kill
				x = int(x)	
			\end{tabbing}
		}
		будет выполняться следующим образом. Сначала выполнится выражение стоящее справа, затем порожденный им объект, сохранится в некоторой области памяти, вообще говоря, отличающейся от того участка, на который указывал $x$ ранее. В заключении ссылка $x$ связывается с этим новым объектом.
		
		Для того чтобы справиться с утечкой памяти старый объект удаляется сборщиком мусора, если на него больше нет ссылок.
		
		При таком подходе тип объекта строго определен, но появляется у ссылки только в момент выполнения программы.
		% СНОСКА: В дальнейшем под переменными будут подразумеваться ссылки.
		
		
		
	\subsection*{Отличия языков программирования Python\,2 и Python\,3}
	
		При развитии и улучшении языков программирования часто бывает необходимо кардинально изменить концепции привычных вещей. При таком переходе теряется совместимость старых и новых версий языка. Так произошло с Python2 и Python3. 
		
		\begin{example}
		
		Функция input() в этих версиях языка ведет себя по разному. Так в Python2 выражение
		\texttt{
			\begin{tabbing}
				\hspace{8mm}\=\hspace{2cm}\=\kill
				x = input('5 + 3')	
			\end{tabbing}
		}
		вычислит значение суммы и сохранит его в $x$.
		В Python3 это же выражение выведет на экран
		\texttt{
			\begin{tabbing}
				\hspace{8mm}\=\hspace{2cm}\=\kill
				5 + 3	
			\end{tabbing}
		}
		а затем считает данные с клавиатуры и сохранит их в $x$ в виде строки.
		Это бывает очень удобно. Например, можно вывести своеобразное приглашение: "Введите $x$". 	
			
		\end{example}
		
	\subsection*{Обмен двух переменных значениями}
	
	% Как правильно программировать, ссылки на статьи.
	% Марк Лутц <<Карманный справочник>>
	% stepic.org
	% cheakio.org - решение задач в игровой форме
	% pythontutor.ru
	% Саммерфилд 2 книги
	% Марк Пилгрим <<Drive into the Python>>
	
	Допустим, есть две переменные
	\texttt{
		\begin{tabbing}
			\hspace{8mm}\=\hspace{2cm}\=\kill
			x = 3\\
			y = 7	
		\end{tabbing}
	}
	и необходимо произвести обмен их значениями. Эта задача однако не так тривиальна, как кажется. Например, такое решение 
	\texttt{
		\begin{tabbing}
			\hspace{8mm}\=\hspace{2cm}\=\kill
			x = y\\
			y = x	
		\end{tabbing}
	}
	является неверным, так как после выполнения $x=y$ теряется ссылка на объект <<3>> и $х$ начинает указывать на <<7>>.
	
	В решении данной задачи удобно воспользоваться аналогией. Так, чтобы поменять содержимое стакана с водой и кружки с молоком друг между другом, можно воспользоваться третьим сосудом.
	
	Этот алгоритм можно реализовать следующим образом.
	\texttt{
		\begin{tabbing}
			\hspace{8mm}\=\hspace{2cm}\=\kill
			tmp = x\\
			x = y\\
			y = x	
		\end{tabbing}
	}
	Теперь нет потери значения $x$, так как оно предварительно сохранено в переменную $tmp$. Это классический пример обмена значениями переменных через третью.
	% СНОСКА алгоритм с двумя временными переменными
	
	\subsection*{Кортеж переменных}
	
	В языке Python существует элегантное решение этой задачи, основанное на использовании специальной структуры данных \emph{кортежа переменных}.
	
	Кортеж синтаксически представляет собой набор данных или переменных разделенных запятыми. Обычно он окружается круглыми скобками. Например,
	\texttt{
		\begin{tabbing}
			\hspace{8mm}\=\hspace{2cm}\=\kill
			(1,2,3)	
		\end{tabbing}
	}
	является кортежем данных, а
	\texttt{
		\begin{tabbing}
			\hspace{8mm}\=\hspace{2cm}\=\kill
			(x, y, z)
		\end{tabbing}
	}
	кортежем переменных.
	
	Такая структура имеет множество удобных способов применения. Так можно присвоить кортежу переменных кортеж данных.
	\texttt{
		\begin{tabbing}
			\hspace{8mm}\=\hspace{2cm}\=\kill
			(x, y, z) = (1, 2, 3)
		\end{tabbing}
	}	
	% СНОСКА Бывает удобно сразу нескольким переменным присвоить одно значение. Для этого в языках \py и Си есть одна удобная конструкция.
	%\[x=y=z=0\]
	Или обменять значениями две переменные более наглядным способом.
	\texttt{
		\begin{tabbing}
			\hspace{8mm}\=\hspace{2cm}\=\kill
			(x, y) = (y, x)
		\end{tabbing}
	}	
	Более того, используя кортежи есть возможность организовать циклический сдвиг нескольких переменных.
	\texttt{
		\begin{tabbing}
			\hspace{8mm}\=\hspace{2cm}\=\kill
			(x, y, z) = (z, x, y)
		\end{tabbing}
	}	
	\subsection*{Арифметические операции}
	
	Большинство арифметических операций синтаксически реализованы в Python нативным образом, как во многих других языках программирования. Например, операции сложения, умножения и деления выглядят следующим образом.
	\texttt{
		\begin{tabbing}
			\hspace{8mm}\=\hspace{2cm}\=\kill
			x + y\\
			x * y\\
			x / y
		\end{tabbing}
	}
	Однако существуют некоторые особенности. Первая из них связанна с операцией целочисленного деления div. Во многих языках программирования она будет выполнена следующим образом.
	\texttt{
		\begin{tabbing}
			\hspace{8mm}\=\hspace{2cm}\=\kill
			-17 div 10 = -1
		\end{tabbing}
	}
	Однако в соответствии с очевидным уравнением
	\texttt{
		\begin{tabbing}
			\hspace{8mm}\=\hspace{2cm}\=\kill
			x * 10 + r = -17
		\end{tabbing}
	}
	% ссылка на уравнение
	это приведет к тому, что соответствующий остаток будет отрицательным. Это противоречит принятому в математике положению, что остаток от деления по модулю~--- неотрицательное число.
	% СНОСКА Кольца вычетов.
	% Рисунок оси 0039 11:20
	Для решения это проблемы в приведенном выше уравнении можно уменьшить $x$ на единицу. Тогда целая часть и остаток от деления будут следующими.
	\texttt{
		\begin{tabbing}
			\hspace{8mm}\=\hspace{2cm}\=\kill
			-17 div 10 = -2\\
			-17 \% 10 = 3
		\end{tabbing}
	}	
	\subsection*{Цикл с предусловием}
	
	% 0039 13:25
	Для организации циклического выполнения каких-либо действий в Python предусмотрена конструкция цикла с предусловием. Она начинается с зарезервированного слова \emph{while}. Затем записывается условие, при котором будут выполняться действия, записанные в \emph{тело цикла}. После условия необходимо поставить двоеточие и перейти на новую строчку.
	Перед каждой следующей строкой тела цикла делается отступ.
	% СНОСКА Рекомендовано делать отступ, равный четырем пробелам относительно положения слова while.
	Для того чтобы выйти из тела цикла необходимо и достаточно перейти на новую строчку и убрать относительный отступ.
	
	При однократном выполнении  или \emph{итерации} цикла в начале будет проверено условие и если оно выполнено, то начнется последовательное исполнение команд тела цикла и переход к новой итерации по их окончании. 
	
	Существуют также \emph{функции управления циклом}. Например, с помощью функции \emph{break} можно преждевременно выйти из цикла. Как правило это бывает нужно, в случаях проверки дополнительного условия в момент выполнения итерации. Условная конструкция оформляется аналогичным образом: после зарезервированного слова \emph{if} следует само условие, оканчивающееся двоеточием. В случае выполнения условия будут выполнены команды, записанные ниже через отступ.
	% СНОСКА Подробнее об условных конструкциях смотри в лекции \textnumero\,(номер лекции).
	
	Также бывает необходимо при выполнении какого-то условия завершить выполнение текущей итерации и перейти к следующей.
	Для этого используется функция \emph{continue}.
	
	В качестве примера оформления можно рассмотреть псевдокод, реализующий бег человека на стадионе.
	\texttt{
		\begin{tabbing}
			\hspace{8mm}\=\hspace{8mm}\=\hspace{2cm}\=\kill
			while sunny:\\
			\> if trouble:\\
			\> \> break\\
			\> run 100 metres\\
			\> if difficult:\\
			\> \> continue\\
			\> run extremelly last 100 metres
		\end{tabbing}
	}
	При таком алгоритме действий человек будет бегать, пока на улице стоит хорошая погода. В случае форс-мажора, например, травмы, бег прекращается. Но если ничего не произошло, то спортсмен будет бежать трусцой 100 метров. Затем, если он устал, снова проверит погоду, оценит, все ли в порядке, пробежит еще 100 метров трусцой, т.\,е. перейдет к следующей итерации цикла. Если же человек полон сил, то последние 100 метров круга он пробежит так быстро, как сможет.
	
	Примером настоящего кода может служить следующая программа.
	\texttt{
		\begin{tabbing}
			\hspace{8mm}\=\hspace{8mm}\=\hspace{2cm}\=\kill
			x = int(input())\\
			while x < 100:\\
			\> s += x\\
			\> if x \% 7 == 0:\\
			\> \> break\\
			\> x += 5\\
			else: \\
			\> print(x)\\
			print(s)
		\end{tabbing}
	}
	% Вычислительная математика
	Она выполняет подсчет суммы арифметической прогрессии с разностью, равной 5,  начиная с введенного пользователем числа (которое приводится к целочисленному типу). Подсчет суммы производится пока члены прогрессии меньше 100. Строка
	\texttt{
		\begin{tabbing}
			\hspace{8mm}\=\hspace{2cm}\=\kill
			s += x
		\end{tabbing}
	}
	означает, что значение переменной $s$  увеличивается на $x$. При этом если текущий просуммированный член прогрессии делится нацело на $7$ (т.е. его остаток при делении на $7$ равен $0$), то осуществляется выход из цикла. Для сравнения равенства двух чисел используется оператор $==$.
	% ? это точно называется оператором?
	В завершении программы на экран выводится значение суммы.
	
	Для того чтобы по окончании работы цикла было понятно, завершился ли он нормально или преждевременно, существует специальное дополнение к конструкции цикла. После тела на уроне while записывается else и ставится двоеточие. Инструкции записанные ниже через отступ будут выполнены только когда будет нарушено условие в заголовке цикла, т.е. когда последний завершится нормально. В приведенном примере на экран выведется последнее значение $x$. Функция break осуществляет полный выход из цикла, а значит, <<команды else>> будут также не выполнены.
	
	% СНОСКА если в конце сторки ставится двоеточие, то следующая строка обязательно начинается с отступа.
	
	В условной конструкции также есть дополнение else. Синтаксически она выглядит так.
	\texttt{
		\begin{tabbing}
			\hspace{8mm}\=\hspace{2cm}\=\kill
			if условие:\\
			\> действие А\\
			else:\\
			\> действие В
		\end{tabbing}
	}

	\subsection*{Генерация~последовательностей.}

	Генерация арифметической прогрессии от нуля до вводимого пользователем $N$, не включая последнего, с разностью, равной $1$.
	\texttt{
		\begin{tabbing}
			\hspace{8mm}\=\hspace{8mm}\=\hspace{2cm}\=\kill
			N = int(input())\\
			i = 0\\
			while i < N:\\
			\> print(i)\\
			\> i += 1
		\end{tabbing}
	}

	Удобно использовать понятие \emph{инварианта цикла}. Это некоторое утверждение, которое является верным в начале каждой итерации цикла. В данном примере оно может быть следующим: <<В момент начала каждой итерации распечатаны все целые числа от $0$ до $N$, не включая последнее>>. 
	
	\subsection*{Вложенные циклы.}
	
	Задача вывода на экран показаний часов с 5 до 12, когда минуты равны 10, 20, 30, 40 и 50.
	\texttt{
		\begin{tabbing}
			\hspace{8mm}\=\hspace{8mm}\=\hspace{2cm}\=\kill
			hour = 5\\
			while hour <= 12:\\
			\> minute = 10\\
			\> while minute <= 50:\\
			\> \> print(hour, minute, sep=':')\\
			\> \> minute += 10\\
			\> hour += 1
		\end{tabbing}
	}
		
	Чтобы не прописывать распечатку пяти показаний для каждого часа, логично использовать еще один, вложенный, цикл, в котором итератор будет пробегаться по значениям минут.
	
	Стоит отметить, что функция break осуществляет выход только из цикла, в чьем теле она записана, т.е. с ее помощью нельзя выйти из вложенного цикла.
	
	
	\subsection*{Функции}
	
	Для того чтобы программа была понятной любому программисту, в том числе и ее автору некоторое время спустя, важно давать функциям и переменным содержательные названия. Например, нельзя понять вне контекста результат работы функции f(a,~b). Однако можно ожидать, что функция max(a,~b) возвратит максимальное из двух чисел $a$ и $b$.
	
	Рассмотрим реализацию последней в качестве примера синтаксиса.
	\texttt{
		\begin{tabbing}
			\hspace{8mm}\=\hspace{8mm}\=\hspace{2cm}\=\kill
			def max(a, b):\\
			\> if a > b:\\
			\> \> return a\\
			\> return b
		\end{tabbing}
	}	
	Заголовок объявления функции начинается с зарезервированного слова \emph{def}. Ввиду того, что Python является интерпретируемым языком, типы данных аргументов функции не указываются, так как они определяются только в момент вызова функции. Поэтому приведенная программа будет одинаково хорошо работать как с целыми, так и с вещественными числами.
	
	В данном примере не используется конструкция с else, так как функция return возвращает результат работы функции (значение переменной $a$ или $b$) и завершает ее выполнение.
	Поэтому выполнится либо return a, либо return b.
	
	Также функция имеет возможность возвращать кортеж. Примером может служить функция, осуществляющая обмен двух переменных значениями.
	\texttt{
		\begin{tabbing}
			\hspace{8mm}\=\hspace{2cm}\=\kill
			def swap(x,y):\\
			\> return y,x
		\end{tabbing}
	}
\end{document}