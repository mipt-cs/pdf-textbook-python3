\documentclass[a4paper,12pt]{article}

%%% Работа с русским языком % для pdfLatex
\usepackage{cmap}					% поиск в~PDF
\usepackage{mathtext} 				% русские буквы в~фомулах
\usepackage[T2A]{fontenc}			% кодировка
\usepackage[utf8]{inputenc}			% кодировка исходного текста
\usepackage[english,russian]{babel}	% локализация и переносы
\usepackage{indentfirst} 			% отступ 1 абзаца

%%% Работа с русским языком % для XeLatex
%\usepackage[english,russian]{babel}   %% загружает пакет многоязыковой вёрстки
%\usepackage{fontspec}      %% подготавливает загрузку шрифтов Open Type, True Type и др.
%\defaultfontfeatures{Ligatures={TeX},Renderer=Basic}  %% свойства шрифтов по умолчанию
%\setmainfont[Ligatures={TeX,Historic}]{Times New Roman} %% задаёт основной шрифт документа
%\setsansfont{Comic Sans MS}                    %% задаёт шрифт без засечек
%\setmonofont{Courier New}
%\usepackage{indentfirst}
%\frenchspacing

%%% Дополнительная работа с математикой
\usepackage{amsfonts,amssymb,amsthm,mathtools}
\usepackage{amsmath}
\usepackage{icomma} % "Умная" запятая: $0,2$ --- число, $0, 2$ --- перечисление
\usepackage{upgreek}

%% Номера формул
%\mathtoolsset{showonlyrefs=true} % Показывать номера только у тех формул, на которые есть \eqref{} в~тексте.

%%% Страница
\usepackage{extsizes} % Возможность сделать 14-й шрифт

%% Шрифты
\usepackage{euscript}	 % Шрифт Евклид
\usepackage{mathrsfs} % Красивый матшрифт

%% Свои команды
\DeclareMathOperator{\sgn}{\mathop{sgn}} % создание новой конанды \sgn (типо как \sin)
\usepackage{csquotes} % ещё одна штука для цитат
\newcommand{\pd}[2]{\ensuremath{\cfrac{\partial #1}{\partial #2}}} % частная производная
\newcommand{\abs}[1]{\ensuremath{\left|#1\right|}} % модуль
\renewcommand{\phi}{\ensuremath{\varphi}} % греческая фи
\newcommand{\pogk}[1]{\!\left(\cfrac{\sigma_{#1}}{#1}\right)^{\!\!\!2}\!} % для погрешностей

% Ссылки
\usepackage{color} % подключить пакет color
% выбрать цвета
\definecolor{BlueGreen}{RGB}{49,152,255}
\definecolor{Violet}{RGB}{120,80,120}
% назначить цвета при подключении hyperref
\usepackage[unicode, colorlinks, urlcolor=blue, linkcolor=blue, pagecolor=blue, citecolor=blue]{hyperref} %синие ссылки
%\usepackage[unicode, colorlinks, urlcolor=black, linkcolor=black, pagecolor=black, citecolor=black]{hyperref} % для печати (отключить верхний!)


%% Перенос знаков в~формулах (по Львовскому)
\newcommand*{\hm}[1]{#1\nobreak\discretionary{}
	{\hbox{$\mathsurround=0pt #1$}}{}}

%%% Работа с картинками
\usepackage{graphicx}  % Для вставки рисунков
\graphicspath{{images/}{images2/}}  % папки с картинками
\setlength\fboxsep{3pt} % Отступ рамки \fbox{} от рисунка
\setlength\fboxrule{1pt} % Толщина линий рамки \fbox{}
\usepackage{wrapfig} % Обтекание рисунков и таблиц текстом
\usepackage{multicol}

%%% Работа с таблицами
\usepackage{array,tabularx,tabulary,booktabs} % Дополнительная работа с таблицами
\usepackage{longtable}  % Длинные таблицы
\usepackage{multirow} % Слияние строк в~таблице
\usepackage{caption}
\captionsetup{labelsep=period, labelfont=bf}

%%% Оформление
\usepackage{indentfirst} % Красная строка
%\setlength{\parskip}{0.3cm} % отступы между абзацами
%%% Название разделов
\usepackage{titlesec}
\titlelabel{\thetitle.\quad}
\renewcommand{\figurename}{\textbf{Рис.}}		%Чтобы вместо figure под рисунками писал "рис"
\renewcommand{\tablename}{\textbf{Таблица}}		%Чтобы вместо table над таблицами писал Таблица

%%% Теоремы
\theoremstyle{plain} % Это стиль по умолчанию, его можно не переопределять.
\newtheorem{theorem}{Теорема}[section]
\newtheorem{proposition}[theorem]{Утверждение}

\theoremstyle{definition} % "Определение"
\newtheorem{definition}{Определение}[section]
\newtheorem{corollary}{Следствие}[theorem]
\newtheorem{problem}{Задача}[section]

\theoremstyle{remark} % "Примечание"
\newtheorem*{nonum}{Решение}
\newtheorem{zamech}{Замечание}[theorem]

%%% Правильные мат. символы для русского языка
\renewcommand{\epsilon}{\ensuremath{\varepsilon}}
\renewcommand{\phi}{\ensuremath{\varphi}}
\renewcommand{\kappa}{\ensuremath{\varkappa}}
\renewcommand{\le}{\ensuremath{\leqslant}}
\renewcommand{\leq}{\ensuremath{\leqslant}}
\renewcommand{\ge}{\ensuremath{\geqslant}}
\renewcommand{\geq}{\ensuremath{\geqslant}}
\renewcommand{\emptyset}{\varnothing}

%%% Для лекций по инфе
\usepackage{tikz}  
\usetikzlibrary{graphs}
\usepackage{alltt}
\newcounter{infa}[section]
\newcounter{num}
\definecolor{infa}{rgb}{0, 0.2, 0.89}
\definecolor{infa1}{rgb}{0, 0.3, 1}
\definecolor{grey}{rgb}{0.5, 0.5, 0.5}
\newcommand{\tab}{\ \ \ }
\newcommand{\com}[1]{{\color{grey}\# #1}}
\newcommand{\num}{\addtocounter{num}{1}\arabic{num}\tab}
\newcommand{\defi}{{\color{infa}def}}
\newcommand{\globali}{{\color{infa}global}}
\newcommand{\ini}{{\color{infa}in}}
\newcommand{\rangei}{{\color{infa}range}}
\newcommand{\fori}{{\color{infa}for}}
\newcommand{\ifi}{{\color{infa}if}}
\newcommand{\elsei}{{\color{infa}else}}
\newcommand{\printi}{{\color{infa1}print}}
\newcommand{\enumeratei}{{\color{infa1}enumerate}}
\newcommand{\maxi}{{\color{infa}max}}
\newcommand{\classi}{{\color{infa}class}}
\newcommand{\returni}{{\color{infa}return}}
\newcommand{\elifi}{{\color{infa}elif}}
\newcommand{\seti}{{\color{infa}set}}
\newcommand{\noti}{{\color{infa}not}}
\newcommand{\dicti}{{\color{infa}dict}}
\newcommand{\zipi}{{\color{infa}zip}}
\newcommand{\chri}{{\color{infa}chr}}
\newcommand{\ordi}{{\color{infa}ord}}
\newcommand{\leni}{{\color{infa}len}}
\newcommand{\deli}{{\color{infa}del}}
\newcommand{\sortedi}{{\color{infa}sorted}}
\newcommand{\keyi}{{\color{infa}key}}
\newcommand{\lambdai}{{\color{infa}lambda}}
\newcommand{\inti}{{\color{infa}int}}
\newcommand{\inputi}{{\color{infa}input}}
\newcommand{\isi}{{\color{infa}is}}
\newcommand{\Nonei}{{\color{infa}None}}
\newcommand{\whilei}{{\color{infa}while}}
\newcommand{\andi}{{\color{infa}and}}
\newcommand{\fromi}{{\color{infa}from}}
\newcommand{\importi}{{\color{infa}import}}
\newcommand{\continuei}{{\color{infa}continue}}
\newcommand{\mapi}{{\color{infa}map}}
\newcommand{\Falsei}{{\color{infa1}False}}
\newcommand{\listi}{{\color{infa}list}}
\newcommand{\Truei}{{\color{infa1}True}}
\newcommand{\mini}{{\color{infa1}min}}


\newenvironment{infa}[1]{
	
	\vspace{0.5cm}
	\addtocounter{infa}{1}%
	\noindent{\large \textbf{Программа №\thesection.\arabic{infa}.}\ \textbf{#1}}%
	\begin{alltt}%
	}{\end{alltt}
	\setcounter{num}{0}
	\vspace{0.1cm}}
\newenvironment{infanoname}{
	
	%\vspace{0.5cm}
	%\addtocounter{infa}{1}%
	%\noindent{\large \textbf{Программа №\thesection.\arabic{infa}.}\ \textbf{#1}}%
	\begin{alltt}%
	}{\end{alltt}
	\setcounter{num}{0}
	\vspace{0.1cm}}

\usepackage{animate} % Для добавления гифок
\usepackage{xmpmulti}
%Пример кода:
%\begin{infa}{Поразрядная сортировка}
%	\ \num \defi count_sort(a):\tab \com{определяет нашу функцию}
%	\ \num \tab m = \maxi(a)+1
%	\ \num \tab q = [0]*m
%	\ \num \tab \fori x \ini a:
%	\ \num \tab \tab q[x] += 1
%	\ \num \tab pos = 0
%	\ \num \tab \fori x \ini q:
%	\ \num \tab \tab \fori i \ini \rangei(q[x]):
%	\ \num \tab \tab \tab a[pos] = x
%	\num \tab \tab \tab pos += 1
%\end{infa}

\usepackage[left=1.27cm,right=1.27cm,top=1.27cm,bottom=2cm]{geometry}
%\hbox to\textwidth{команда колонтитула}
\begin{document}
\newcounter{lec}
\newcommand{\lec}[1]{\addtocounter{lec}{1} \setcounter{section}{0}%
\begin{center}
{\LARGE ЛЕКЦИЯ \arabic{lec}%
\vspace{2mm}%

\textbf{#1}%
}
\end{center}
}
\newpage
\
\setcounter{lec}{19}
\lec{Задачи на графы}
\section{Поиск минимального остовного дерева}
Минимальное остовное дерево (для связного графа) --- остовное дерево минимального суммарного веса. \\
Остовных деревьев может быть несколько. Есть 2 алгоритма поиска минимального остовного дерева: Прима и Краскала. 
Оба алгоритмы жадные. В первом случае мы выхватываем ребро, которое короче. Связность появится в последний момент. Во втором случае мы постоянно поддерживаем связанность.
Условие: связный неориентированный граф.
\subsection{Алгоритм Краскала}
\subsubsection{Алгоритм}
\begin{enumerate}
\item Упорядочиваем все ребра по возрастанию веса (не убыванию).
\item Далее добавляем ребро по порядку к каркасу (вершины добавляются вместе с ребром), если оно не образует цикл с уже имеющимися  ранее ребрами.
\end{enumerate}
Подробная визуализация разобрана на \href{https://ru.wikipedia.org/wiki/%D0%90%D0%BB%D0%B3%D0%BE%D1%80%D0%B8%D1%82%D0%BC_%D0%9A%D1%80%D0%B0%D1%81%D0%BA%D0%B0%D0%BB%D0%B0}{википедии}.
\subsubsection{Реализация}
\begin{infa}{Реализация алгоритма Краскала}
\ \num N, M = [\inti(x) \fori x \ini \inputi().split()]
\ \num edges = []
\ \num \fori i \ini \rangei(M): 
\ \num \tab v1, v2, weight = \mapi(\inti, \inputi().split())
\ \num \tab edges.append((weight, v1, v2)) \com{Сначала будем добавлять вес}
\ \num edges.sort()
\ \num comp = \listi(range(N))
\ \num tree = []
\ \num tree_weight = 0
\num \fori weight, v1, v2 \ini edges:
\num \tab \ifi comp[v1] != comp[v2]:
\num \tab \tab tree.append((v1, v2))
\num \tab \tab tree_weight += weight
\num \tab \tab \fori i \ini \rangei(N):
\num \tab \tab \tab \ifi comp[i] == comp[v2]:
\num \tab \tab \tab \tab comp[i] = comp[v1]
\end{infa}
Сложность алгоритма $O(M \log M + M\cdot N)$.\\
Построчный комментарий кода:\\
1) Вводим количество вершин и ребер.\\
2) Создаем список для ребер.\\
4) Вводим ребра --- вершины, которые они соединяют и вес.\\
5) Добавляем в список ребер.\\
6) Сортируем по весу ребер.\\
7) Создаем список ребер, чтобы работать с индексами и проверять, из одной они компоненты связанности или нет.\\
8) Сам массив ребер, образующих остовное дерево.\\
9) Вес графа.\\
10) Пока не прошли по всем ребрам.\\
11) Если вершины не принадлежат одной компоненте связанности.\\
12) Добавляем ребро к остовному дереву.\\
13) Добавляем в общий вес вес нового ребра.\\
14) Проходим по всем вершинам.\\
15-16) Если мы встречаем аналогичную второй вершине компоненту связанности,то перекрашиваем ее в компоненту связанности первой. Так свяжем дерево.\\






















\subsection{Алгоритм Прима}
%Есть множество уже присоединенных к остовному дереву.
\subsubsection{Алгоритм}
\begin{enumerate}
	\item Выбираем произвольную вершину (она и есть заготовка для дерева).
	\item Добавляем к каркасу ребро минимального веса среди ребер, инцидентных какой-либо вершине каркаса и вершине не из каркаса.
\end{enumerate}
\subsubsection{Реализация}
\texttt{dist[i]} --- минимальный вес  ребра, которым можно присоединить вершину i к каркасу.
\begin{infa}{Реализация}
\ \num INF = 10**9 \com{Введем условную бесконечность}
\ \num dist = [INF]*N \com{W[i][j] - вес ребра ij, который равен +бесконечность,\\ \phantom{\ 2\tab }если i не смежна j} 
\ \num dist[0] = 0
\ \num used = [\Falsei]*N
\ \num used[0] = \Truei
\ \num tree = []
\ \num tree_weight = 0
\ \num \fori i \ini \rangei(N):
\ \num \tab min_d = INF
\num \tab \fori j \ini \rangei(N):
\num \tab \tab \ifi \noti used[j] \andi dist[j] < min_d:
\num \tab \tab \tab min_d = dist[j]
\num \tab \tab \tab u = j
\num \tab tree.append((i, u))
\num \tab tree_weight += min_d
\num \tab used[u] = \Truei
\num \tab \fori v \ini \rangei(N):
\num \tab \tab dist[v] = \mini(dist[v], W[u][v])
\end{infa}
Асимптотика алгоритма: $O(N^2)$. С помощью кучи можно ускорить до $O((M+N)\log N)$.
\section{Алгоритм построения Гамильтонова цикла}
Гамильтонов цикл~--- цикл, проходящий через все вершины по одному разу.
Гамильтонов путь~--- путь, проходящий через все вершины по одному разу.\\
Это NP -- сложная задача, т.к. решение задачи находится перебором, $O(N!)$.
\begin{infa}{Реализация}
\ \num visited = [\Falsei]*N
\ \num path = []
\ \num \defi hamilton(curr):
\ \num \tab path.append(curr)
\ \num \tab \ifi \leni(path) == N:
\ \num \tab \tab \ifi A[path[-1]][path[0]] == 1: \com{A - таблица смежности}
\ \num \tab \tab \tab \returni \Truei
\ \num \tab \tab \elsei:
\ \num \tab \tab \tab path.pop()
\num \tab \tab \tab \returni \Falsei
\num \tab visited[curr] = \Truei
\num \tab \fori next \ini \rangei(N):
\num \tab \tab \ifi A[curr][next] == 1 \andi \noti visited[next]:
\num \tab \tab \tab \ifi hamilton(next):
\num \tab \tab \tab \tab \returni \Truei 
\num \tab visited[curr] = \Falsei
\num \tab path.pop()
\num \tab \returni \Falsei
\end{infa}
























\begin{center}
	\vfill \emph{{\small Г. С. Демьянов, \href{https://vk.com/id37346992}{VK}\\
С. С. Клявинек, \href{https://vk.com/id85132547}{VK}
}}
\end{center}






\end{document} 